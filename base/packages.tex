\usepackage[utf8]{inputenc}

\usepackage{amscd}

\usepackage{amsfonts}

\usepackage{amsmath}

\usepackage{amssymb}

\usepackage{amsthm}

% Manages culturally-determined typographical rules for a
% wide range of languages.
\usepackage[spanish, english]{babel}

% Defines commands to access bold math symbols
% \usepackage{bm}

% Offers customization of captions in floating environments
% \usepackage{caption}

\usepackage{euscript}

% Defines floating objects such as figures and tables
% \usepackage{float}

% Allows the user to select font encodings.
\usepackage[T1]{fontenc}

% Creates regions that can break across pages
\usepackage{framed}

% Reduces horizontal margin
% \usepackage{fullpage}

% Provides an easy and flexible user interface to customize page layout,
% implementing auto-centering and auto-balancing mechanisms.
\usepackage[lmargin=2.5cm, rmargin=2.5cm, tmargin=3.0cm, bmargin=3.0cm]{geometry}

% Enhances graphics package
% \usepackage{graphicx}
\usepackage[dvips]{graphicx}

\usepackage{hyperref}

% Translates various standard and other input encodings.
\usepackage[utf8]{inputenc}

% Enhances LaTeX cross-referencing
\usepackage[nameinlink]{cleveref}

\usepackage{latexsym}

\usepackage{mathtools}