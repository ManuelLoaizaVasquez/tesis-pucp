\pretolerance=20000\tolerance=30000
\setlength{\headheight}{14.61858pt}
\selectlanguage{spanish}
\pagestyle{fancy}
\chapter[Primer cap\'itulo]
{Incluir el t\'itulo completo del primer cap\'itulo}
\label{Chapter01}
\vspace*{1cm}
\begin{center}
  \begin{minipage}{12cm}
    \texttt{
      \baselineskip 10pt
      \begin{flushright}
        Puede introducir una cita \\
        de su elecci\'on. \\
        \sf{Autor de la cita.}
      \end{flushright}
    }
    \textsl{
      \baselineskip 10pt
      Escribir aqu\'i un resumen del primer cap\'itulo en el cual
      se describe el contenido, el objetivo y las principales
      referencias que se utilizaron.
      Se incluye una adecuada justificaci\'on de por qu\'e es necesario
      el desarrollo de este cap\'itulo.
    }
  \end{minipage}
\end{center}

\section{Presentaci\'on de la tesis}
Para una adecuada redacci\'on de la tesis es \'util recordar algunos aspectos
formales en la presentaci\'on final.
Por ejemplo, no s\'olo es preferible usar el lenguaje formal,
sino tambi\'en tomar en cuenta que el estilo objetivo de los documentos
acad\'emicos prioriza la redacci\'on en tercera persona:
<<los autores consideran>> \, o <<se considera>>.
En este contexto, una vez terminada la redacci\'on formal es necesario
realizar una revisi\'on final del trabajo, con una relectura cr\'itica de
todo el manuscrito para eliminar las posibles incoherencias en la redacci\'on,
corregir y presentar el adecuado uso de los signos de puntuaci\'on, revisar
el uso correcto de las reglas de la ortograf\'ia, etc. 
En esta etapa, vale la pena revisar los aspectos formales tales como la forma
adecuada de citar las referencias bibliogr\'aficas, entre otros.

\subsection{Ortograf\'ia}
En la redacci\'on correcta de los p\'arrafos que componen la disertaci\'on
final, se debe buscar la claridad y coherencia del texto, tomando en cuenta
no solo al jurado, sino tambi\'en a cualquier  lector interesado en la tesis.
Para lograr este objetivo es conveniente evitar los errores ortogr\'aficos y
el mal uso de los signos de puntuaci\'on.
Al respecto, la peque\~na gu\'ia \cite{Milnor1978} es de gran utilidad en la
revisi\'on del estilo y la ortograf\'ia del texto matem\'atico;
esta guia, entre otras cosas, incluye con claridad el uso adecuado de las
may\'usculas. De modo similar, en el libro \cite{BEM2022} se recomienda
saber googlear antes de preguntar.
Finalmente, se sugiere analizar y leer con detalle \cite{Sperner1928},
en donde se presentan algunas recomendaciones pr\'acticas y apropiadas para
evitar algunos errores frecuentes.

\subsection{Orden de presentaci\'on}
Las siguientes componentes son las que se deber\'ian incluir como parte de la
tesis. Algunas de ellas son opcionales.

\begin{enumerate}
  \item Car\'atula.
  \item Hoja de presentaci\'on y aprobaci\'on.
  \item Resumen ejecutivo (m\'aximo 500 palabras).
  \item Dedicatoria (opcional).
  \item \'Indice o Contenido.
  \item Lista de figuras (opcional).
  \item Agradecimientos (opcional).
  \item Introducci\'on.
  \item Cuerpo de la tesis.
  \item Conclusiones.
  \item Ap\'endices (opcional).
  \item Bibliograf\'ia.
  \item \'Indice alfab\'etico (opcional).
\end{enumerate}

\subsection{Numeraci\'on y m\'argenes}
En la descripci\'on anterior,
los siete primeros \'items tienen que ocupar \textbf{una p\'agina}
(con excepci\'on del contenido),
las cuales deben ser numeradas en romano y con min\'usculas.
La numeraci\'on ar\'abiga empieza en la introducci\'on.

\par

Respecto a los m\'argenes, la presentaci\'on final del trabajo de tesis debe
hacerse en el formato A\(4\) (\(210 \times 297\) mm), a una sola cara en letra
de tama\~no \(12\), en espacio simple con margen superior e inferior de \(2.5\)
cm y margen en los lados de \(3\) cm.
Cada uno de estos requerimientos formales,
para la versi\'on digital del documento final de la disertaci\'on,
se encuentran presentes en la plantilla.

\subsection{Tablas y figuras}
Las tablas y figuras deben estar numeradas y citadas en el desarrollo del
texto. Adem\'as, se incorporan dentro  del texto y no al final del cap\'itulo
o en ap\'endices. Para ilustrar esta idea, a continuaci\'on se presenta
\mbox{Figura \ref{LogoAntiguo}} que incluye una tabla con algunos ejemplos del
uso incorrecto de las may\'usculas dentro de la literatura matem\'atica.
Este ejemplo se incluye en la lista de figuras autom\'aticamente.
\begin{figure}[H]
  \centering
  \includegraphics[width=14.5cm]{images/2020-pucp-logo.png}
  \caption{Tabla tomada de \cite{EW2012}}\label{LogoAntiguo}
\end{figure}

\subsection{Cap\'itulos y ap\'endices}
Los cap\'itulos se enumeran con un n\'umero ar\'abigo y se recomienda incluir
una minip\'agina con una descripci\'on del respectivo cap\'itulo.
Los ap\'endices se ordenan con letras may\'usculas y deben aparecer en el
contenido.

\subsection{Bibliograf\'ia}
Se recomienda utilizar BibTeX con el estilo \texttt{amsalpha},
el cual produce etiquetas usando el nombre del autor y el a\~no de la
publicaci\'on. El estilo \texttt{amsplain} genera números positivos.

